% This is samplepaper.tex, a sample chapter demonstrating the
% LLNCS macro package for Springer Computer Science proceedings;
% Version 2.20 of 2017/10/04

\documentclass[runningheads]{llncs}

\usepackage{graphicx}
\usepackage{hyperref}
% Used for displaying a sample figure. If possible, figure files should
% be included in EPS format.
%
% If you use the hyperref package, please uncomment the following line
% to display URLs in blue roman font according to Springer's eBook style:
%\renewcommand\UrlFont{\color{blue}\rmfamily}
% TODO: check why the command above gives an error

% Packages for computer code
\usepackage{algorithm}
\usepackage{algpseudocode}
% Package for multiline comments
\usepackage{verbatim}

\begin{document}

\title{Additional symmetry-breaking for constrained guillotine 2D packing\thanks{TODO: add project.}}

%\titlerunning{Abbreviated paper title}
% If the paper title is too long for the running head, you can set
% an abbreviated paper title here

\author{Henrique Becker\inst{1}\orcidID{0000-0003-3879-2691} \and
Olinto Araujo\inst{1,2}\orcidID{0000-0003-1136-5032} \and
Luciana S. Buriol\inst{1}\orcidID{0000-0002-9598-5732}}

\authorrunning{H. Becker et al.}
% First names are abbreviated in the running head.
% If there are more than two authors, 'et al.' is used.

\institute{
  Federal University of Rio Grande do Sul (UFRGS), Postgraduate on Computer Science Program, Av. Bento Gonçalves, 9500, Porto Alegre, RS, Brazil\\
  \email{ppgc@inf.ufrgs.br}\\
  \url{http://www.inf.ufrgs.br/ppgc/}\\
  \and
  Federal University of Santa Maria (UFSM), Postgraduate on Computer Science Program, Av. Roraima, 1000, Santa Maria, RS, Brazil\\
  \email{ppgcc@ufsm.br}\\
  \url{https://www.ufsm.br/cursos/pos-graduacao/santa-maria/ppgcc/}\\
}

\maketitle

\begin{abstract}
%The abstract should briefly summarize the contents of the paper in
%150--250 words.

\keywords{TODO \and CHECK KEYWORDS \and USED IN OTHER ARTICLES.}
\end{abstract}

%\caption{Table captions should be placed above the tables.}
%\begin{equation}
%x + y = z
%\end{equation}
%\caption{A figure caption is always placed below the illustration.}
 
\section{Introduction}

% TODO: check the format of a random sample of 20 papers of IPCO, biased for the last editions
% the informal description of the problem?

\subsection{Problem definition and notation}
% TODO: should we group every definition here? or definitions should appear close to their respective theorems?

\subsection{Prior Work}

% use the prior work to make reference to proofs we will make reference
% cite nicos:1977 for the proof of being able to cut just on the first half of the plate, and every cut always containing at least one piece border segment
% already make use of the notation

\section{Our contributions}

to simplify the writing, we write the theorems and proofs for an specific dimension (often length), and ommit the symmetric proof for the other dimension

\subsection{Second child shortening}

% definition: a restricted set of linear combinations 
% definition: normalized packing (borrowed from nicos), maybe a definition for normalized cut and normalized packing too?
% lemma: if a plate dimension is reduced to a discretized size, no normlized packing will be lost
% definition: creates an adjective like shortened/normalized for the plate size
% remark: this restricted set is not symmetric
% example: of the above remark
% corollary: the second child of a normalized cut over a normalized plate may be non-normalized

\begin{theorem}
\end{theorem}
\begin{proof}
\end{proof}

\subsection{Our changes to Furini's model}

%in the original model
%a cut always creates two childs
%the first child always can fit some piece
%the second child may or may not be able to fit a piece

\begin{theorem}
\end{theorem}
\begin{proof}
\end{proof}

\section{Experimental results}
\section{Conclusions}

%\begin{theorem}
%This is a sample theorem. Definitions, lemmas, propositions, and corollaries are styled the same way.
% the environments 'definition', 'lemma', 'proposition', 'corollary',
% 'remark', and 'example' are defined in the LLNCS documentclass as well.
%\end{theorem}
%\begin{proof}
% after a theorem follows a proof
%\end{proof}


% ---- Bibliography ----
%
% BibTeX users should specify bibliography style 'splncs04'.
% References will then be sorted and formatted in the correct style.
\bibliographystyle{splncs04}
\bibliography{revised_furini}

\begin{comment}
\begin{theorem}
Every non-trivial linear combination of the vector \emph{s} with nonnegative coefficients restricted by the pieces demand and the value smaller-than-or-equal-to 
\end{theorem}

\begin{algorithm}[!htb]
\caption{}
\begin{algorithmic}[1]
\Procedure{rlc}{$S, s_1, \dots, s_n, d_1, \dots, d_n$}
  % TODO: get better emptyset
  \State \(C \gets \emptyset\) 

  \For{\(i \gets 1\) to \(n\)}
    \State \(C^\prime \gets \emptyset\)
    \For{\(y \in C\)}
      \For{\(q \gets 1\) to \(d_i\)}
        \State \(C^\prime \gets C^\prime \cup \{y + s_i \times q\}\)
      \EndFor
    \EndFor
    \State \(C \gets C \cup C^\prime\)

    \For{\(q \gets 1\) to \(d_i\)}
      \State \(C \gets C \cup \{s_i \times q\}\)
    \EndFor
  \EndFor

  \State \textbf{return}~\(C\)
\EndProcedure
\end{algorithmic}
\end{algorithm}

% TODO: check if there is a way to supress the block ends and use only
% identation to demark blocks
% TODO: ASK OLINTO IF WE SHOULD USE BRACKETS INSTEAD OF SUBSCRIPT FOR
% INDEXING VECTORS
\begin{algorithm}[!htb]
\caption{}
\begin{algorithmic}[1]
\Procedure{rlc}{$S, s_1, \dots, s_n, d_1, \dots, d_n$}
  \State \(b_1, \dots, b_S \gets\) false\(,\dots,\)false
  
  \For{\(i \gets 1\) to \(n\)}%\label{begin_trivial_bounds}\Comment{Stores one-item solutions}

    \For{\(y \gets S\) to \(1\) by step \(-1\)}
      \If{\(b_y\)}
        \For{\(q \gets 1\) to \(d_i\)}
	  \State \(y^\prime \gets y + s_i \times q\)
	  \If{\(y^\prime \leq S\)}
            \State \(b_{y^\prime} \gets\) true
	  \EndIf
	\EndFor
      \EndIf
    \EndFor

    \For{\(q \gets 1\) to \(d_i\)}
      \If{\(s_i \times q \leq S\)}
        \State \(b_{s_i \times q} \gets\) true
      \EndIf
    \EndFor
  \EndFor

  \State \textbf{return}~\(b\)
\EndProcedure
\end{algorithmic}
\end{algorithm}
\end{comment}

\end{document}

